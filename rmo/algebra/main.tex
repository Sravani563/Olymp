\documentclass[journal,12pt,onecolumn]{IEEEtran}
%
\usepackage{setspace}
\usepackage{gensymb}
%\doublespacing
\singlespacing

%\usepackage{graphicx}
%\usepackage{amssymb}
%\usepackage{relsize}
\usepackage[cmex10]{amsmath}
%\usepackage{amsthm}
%\interdisplaylinepenalty=2500
%\savesymbol{iint}
%\usepackage{txfonts}
%\restoresymbol{TXF}{iint}
%\usepackage{wasysym}
\usepackage{amsthm}
%\usepackage{iithtlc}
\usepackage{mathrsfs}
\usepackage{txfonts}
\usepackage{stfloats}
\usepackage{bm}
\usepackage{cite}
\usepackage{cases}
\usepackage{subfig}
%\usepackage{xtab}
\usepackage{longtable}
\usepackage{multirow}
%\usepackage{algorithm}
%\usepackage{algpseudocode}
\usepackage{enumitem}
\usepackage{mathtools}
\usepackage{tikz}
\usepackage{circuitikz}
\usepackage{verbatim}
%\usepackage{tfrupee}
\usepackage[breaklinks=true]{hyperref}
%\usepackage{stmaryrd}
\usepackage{tkz-euclide} % loads  TikZ and tkz-base
\usetkzobj{all}
\usepackage{listings}
    \usepackage{color}                                            %%
    \usepackage{array}                                            %%
    \usepackage{longtable}                                        %%
    \usepackage{calc}                                             %%
    \usepackage{multirow}                                         %%
    \usepackage{hhline}                                           %%
    \usepackage{ifthen}                                           %%
  %optionally (for landscape tables embedded in another document): %%
    \usepackage{lscape}     
\usepackage{multicol}
\usepackage{chngcntr}
%\usepackage{enumerate}

%\usepackage{wasysym}
%\newcounter{MYtempeqncnt}
\DeclareMathOperator*{\Res}{Res}
%\renewcommand{\baselinestretch}{2}
\renewcommand\thesection{\arabic{section}}
\renewcommand\thesubsection{\thesection.\arabic{subsection}}
\renewcommand\thesubsubsection{\thesubsection.\arabic{subsubsection}}

\renewcommand\thesectiondis{\arabic{section}}
\renewcommand\thesubsectiondis{\thesectiondis.\arabic{subsection}}
\renewcommand\thesubsubsectiondis{\thesubsectiondis.\arabic{subsubsection}}

% correct bad hyphenation here
\hyphenation{op-tical net-works semi-conduc-tor}
\def\inputGnumericTable{}                                 %%

\lstset{
%language=C,
frame=single, 
breaklines=true,
columns=fullflexible
}
%\lstset{
%language=tex,
%frame=single, 
%breaklines=true
%}

\begin{document}
%


\newtheorem{theorem}{Theorem}[section]
\newtheorem{problem}{Problem}
\newtheorem{proposition}{Proposition}[section]
\newtheorem{lemma}{Lemma}[section]
\newtheorem{corollary}[theorem]{Corollary}
\newtheorem{example}{Example}[section]
\newtheorem{definition}[problem]{Definition}
%\newtheorem{thm}{Theorem}[section] 
%\newtheorem{defn}[thm]{Definition}
%\newtheorem{algorithm}{Algorithm}[section]
%\newtheorem{cor}{Corollary}
\newcommand{\BEQA}{\begin{eqnarray}}
\newcommand{\EEQA}{\end{eqnarray}}
\newcommand{\define}{\stackrel{\triangle}{=}}

\bibliographystyle{IEEEtran}
%\bibliographystyle{ieeetr}


\providecommand{\mbf}{\mathbf}
\providecommand{\pr}[1]{\ensuremath{\Pr\left(#1\right)}}
\providecommand{\qfunc}[1]{\ensuremath{Q\left(#1\right)}}
\providecommand{\sbrak}[1]{\ensuremath{{}\left[#1\right]}}
\providecommand{\lsbrak}[1]{\ensuremath{{}\left[#1\right.}}
\providecommand{\rsbrak}[1]{\ensuremath{{}\left.#1\right]}}
\providecommand{\brak}[1]{\ensuremath{\left(#1\right)}}
\providecommand{\lbrak}[1]{\ensuremath{\left(#1\right.}}
\providecommand{\rbrak}[1]{\ensuremath{\left.#1\right)}}
\providecommand{\cbrak}[1]{\ensuremath{\left\{#1\right\}}}
\providecommand{\lcbrak}[1]{\ensuremath{\left\{#1\right.}}
\providecommand{\rcbrak}[1]{\ensuremath{\left.#1\right\}}}
\theoremstyle{remark}
\newtheorem{rem}{Remark}
\newcommand{\sgn}{\mathop{\mathrm{sgn}}}
\providecommand{\abs}[1]{\left\vert#1\right\vert}
\providecommand{\res}[1]{\Res\displaylimits_{#1}} 
\providecommand{\norm}[1]{\left\lVert#1\right\rVert}
%\providecommand{\norm}[1]{\lVert#1\rVert}
\providecommand{\mtx}[1]{\mathbf{#1}}
\providecommand{\mean}[1]{E\left[ #1 \right]}
\providecommand{\fourier}{\overset{\mathcal{F}}{ \rightleftharpoons}}
%\providecommand{\hilbert}{\overset{\mathcal{H}}{ \rightleftharpoons}}
\providecommand{\system}{\overset{\mathcal{H}}{ \longleftrightarrow}}
	%\newcommand{\solution}[2]{\textbf{Solution:}{#1}}
\newcommand{\solution}{\noindent \textbf{Solution: }}
\newcommand{\cosec}{\,\text{cosec}\,}
\providecommand{\dec}[2]{\ensuremath{\overset{#1}{\underset{#2}{\gtrless}}}}
\newcommand{\myvec}[1]{\ensuremath{\begin{pmatrix}#1\end{pmatrix}}}
\newcommand{\mydet}[1]{\ensuremath{\begin{vmatrix}#1\end{vmatrix}}}
%\numberwithin{equation}{section}
\numberwithin{equation}{subsection}
%\numberwithin{problem}{section}
%\numberwithin{definition}{section}
\makeatletter
\@addtoreset{figure}{problem}
\makeatother

\let\StandardTheFigure\thefigure
\let\vec\mathbf
%\renewcommand{\thefigure}{\theproblem.\arabic{figure}}
\renewcommand{\thefigure}{\theproblem}
%\setlist[enumerate,1]{before=\renewcommand\theequation{\theenumi.\arabic{equation}}
%\counterwithin{equation}{enumi}


%\renewcommand{\theequation}{\arabic{subsection}.\arabic{equation}}

\def\putbox#1#2#3{\makebox[0in][l]{\makebox[#1][l]{}\raisebox{\baselineskip}[0in][0in]{\raisebox{#2}[0in][0in]{#3}}}}
     \def\rightbox#1{\makebox[0in][r]{#1}}
     \def\centbox#1{\makebox[0in]{#1}}
     \def\topbox#1{\raisebox{-\baselineskip}[0in][0in]{#1}}
     \def\midbox#1{\raisebox{-0.5\baselineskip}[0in][0in]{#1}}

\vspace{3cm}

\title{
%	\logo{
Algebra: Maths Olympiad
%	}
}
\author{ G V V Sharma$^{*}$% <-this % stops a space
	\thanks{*The author is with the Department
		of Electrical Engineering, Indian Institute of Technology, Hyderabad
		502285 India e-mail:  gadepall@iith.ac.in. All content in this manual is released under GNU GPL.  Free and open source.}
	
}	
%\title{
%	\logo{Matrix Analysis through Octave}{\begin{center}\includegraphics[scale=.24]{tlc}\end{center}}{}{HAMDSP}
%}


% paper titles
% can use linebreaks \\ within to get better formatting as desired
%\title{Matrix Analysis through Octave}
%
%
% author names and IEEE memberships
% note positions of commas and nonbreaking spaces ( ~ ) LaTeX will not break
% a structure at a ~ so this keeps an author's name from being broken across
% two lines.
% use \thanks{} to gain access to the first footnote area
% a separate \thanks must be used for each paragraph as LaTeX2e's \thanks
% was not built to handle multiple paragraphs
%

%\author{<-this % stops a space
%\thanks{}}
%}
% note the % following the last \IEEEmembership and also \thanks - 
% these prevent an unwanted space from occurring between the last author name
% and the end of the author line. i.e., if you had this:
% 
% \author{....lastname \thanks{...} \thanks{...} }s
%                     ^------------^------------^----Do not want these spaces!
%
% a space would be appended to the last name and could cause every name on that
% line to be shifted left slightly. This is one of those "LaTeX things". For
% instance, "\textbf{A} \textbf{B}" will typeset as "A B" not "AB". To get
% "AB" then you have to do: "\textbf{A}\textbf{B}"
% \thanks is no different in this regard, so shield the last } of each \thanks
% that ends a line with a % and do not let a space in before the next \thanks.
% Spaces after \IEEEmembership other than the last one are OK (and needed) as
% you are supposed to have spaces between the names. For what it is worth,
% this is a minor point as most people would not even notice if the said evil
% space somehow managed to creep in.



% The paper headers
%\markboth{Journal of \LaTeX\ Class Files,~Vol.~6, No.~1, January~2007}%
%{Shell \MakeLowercase{\textit{et al.}}: Bare Demo of IEEEtran.cls for Journals}
% The only time the second header will appear i/year/1963s for the odd numbered pages
% after the title page when using the twoside option.
% s
% *** Note that you probably will NOT want to include the author's ***
% *** name in the headers of peer review papers.                   ***
% You can use \ifCLASSOPTIONpeerreview for conditional compilation here if
% you desire.




% If you want to put a publisher's ID mark on the page you can do it like
% this:
%\IEEEpubid{0000--0000/00\$00.00~\copyright~2007 IEEE}
% Remember, if you use this you must call \IEEEpubidadjcol in the second
% column for its text to clear the IEEEpubid ma/year/1963rk.



% make the title area
\maketitle



%\tableofcontents

\bigskip

\renewcommand{\thefigure}{\theenumi}
\renewcommand{\thetable}{\theenumi}
%\renewcommand{\theequation}{\theenumi}

%\begin{abstract}
%%\boldmath
%In this letter, an algorithm for evaluating the exact analytical bit error rate  (BER)  for the piecewise linear (PL) combiner for  multiple relays is presented. Previous results were available only for upto three relays. The algorithm is unique in the sense that  the actual mathematical expressions, that are prohibitively large, need not be explicitly obtained. The diversity gain due to multiple relays is shown through plots of the analytical BER, well supported by simulations. 
%
%\end{abstract}
% IEEEtran.cls defaults to using nonbold math in the Abstract.
% This preserves the distinction between vectors and scalars. However,
% if the journal you are submitting to favors bold math in the abstract,
% then you can use LaTeX's standard command \boldmath ast the very start
% of the abstract to achieve this. Many IEEE journals frown on math
% in the abstract anyway.

% Note that keywords are not normally used for peerreview papers.
%\begin{IEEEkeywords}
%Cooperative diversity, decode and forward, piecewise linear
%\end{IEEEkeywords}



% For peer review papers, you can put extra information on the cover
% page as needed:
% \ifCLASSOPTIONpeerreview
% \begin{center} \bfseries EDICS Category: 3-BBND \end{center}
% \fi
%
% For peerreview papers, this IEEEtran command inserts a page break and
% creates the second title. It will be ignored for othesr modes.
%\IEEEpeerreviewmaketitle


%Download python codes using 
%\begin{lstlisting}
%svn co https://github.com/gadepall/school/trunk/ncert/computation/codes
%\end{lstlisting}

\renewcommand{\theequation}{\theenumi}
\begin{enumerate}[label=\arabic*.,ref=\theenumi]
%\begin{enumerate}[label=\arabic*.,ref=\thesubsection.\theenumi]
\numberwithin{equation}{enumi}
\item Find all the primes p and q such that $p^{2} + 7pq + q^{2}$ is the square of an integer.

\item In an acute angle ABC, points D, E, F are located on the sides BC, CA, AB respectively such that 
\begin{align*}
\frac{CD}{CE} = \frac{CA}{CB}, \frac{AE}{AF} = \frac{AB}{AC}, \frac{BF}{BD} = \frac{BC}{BA}
\end{align*}
Prove that AD, BE, CF are the altitude of ABC.

\item The circumference of a circle is divided into eight arcs by a convex quadrilateral ABCD, with
four arcs lying inside the quadrilateral and the remaining four lying outside it. The lengths of
the arcs lying inside the quadrilateral are denoted by p, q, r, s in counter-clockwise direction
starting from some arc. Suppose p + r = q + s. Prove that ABCD is a cyclic quadrilateral.


\item Consider in plane circle $\Gamma$ with center O and a line l not intersecting circle $\Gamma$. Prove that there is a unique point Q on the perpendicular drawn from O to the line l, such that for any point P on the line l, PQ represents the length of the tangent from P to the circle $\Gamma$.

\item Let ABCD be a quadrilateral X and Y be the mid points of AC and BD respectively and the lines through X and Y respectively parllel to BD, AC meet in O. Let P, Q, R, S be the mid points of AB, BC, CD, DA respectively. Prove that
\begin{enumerate}
\item quadrilaterals APOS and APXS have the same area.
\item the areeas is the quadrilateral APOS, BQOP, CROQ, DSOR are all equal. 
\end{enumerate}

\item Find the number of all 5-digit numbers each of which contains the block 15 and is divisible by 15.
 
\item Find the least possible value of a + b, where a, b are positive integers such that 11 divides a + 13b and 13 divides a+11b.

\item A 6 $\times$ 6 square is dissected in to 9 rectangles by lines parallel to its sides such that all these rectangles have integer sides, Prove that there are always two congruent rectangles.

\item Let ABCD be a quadrilateral in which AB is parallel to CD and perpendicular to AD ; AB = 3CD;and the area of the quadrilateral is 4. If a circle can be drawn touching all the sides of the quadrilateral, find its radius.

\item Let a, b, c be the three natural numbers such that $a < b < c$ and ged(c-a, c-b) = 1. Suppose there exists an integer d such that a+d, b+d, c+d from the sides of a right angled triangle. Prove that there exist integers l, m such that 
c + d = $l^2 + m^2$.

\item Find all pairs (a, b) of real numbers such that whenever $\alpha$ is a root of
\begin{align} 
x^2 + ax + b = 0,
\end{align}
$\alpha^2$ - 2 is also a root of the equation.


\item Let ABC be an acute angled triangle; let D, F be the mid-point of BC, AB respectively. Let the perpendicular from F to AC and the perpendicular at B to BC meet in N. Prove that ND is equal to the circum radius of ABC.


\item Find the sum of all 3-digit natural numbers which contain at least one odd digit and at least one even digit.

\item In a book with page number from 1 to 100, some pages are not torn off. The sum of the numbers on the remaining pages is 4949. How many pages are torn off?

\item Let 
\begin{align*}
P_1(x) = ax^2 - bx - c,
\end{align*}
\begin{align*}
P_2(x) = bx^2 - cx - a,
\end{align*}
\begin{align*} 
P_3(x) = cx^2 - ax - b
\end{align*}
be three quadratic polynomials where a, b, c are non-zero real numbers. Suppose there exists a real number 
$\alpha$ such that $P_(\alpha) = P_2(\alpha)= P_3(\alpha)$. Prove that a = b = c

\item Find the number of 4-digit numbers  having non-zero digits and which are divisible by 4 but not 8.
 
\item Find all pairs (x, y) of real numbers such that
\begin{align}
16^{x^2 + y} + 16^{x + y^2} = 1
\end{align}


\item Let a and b positive real numbers such that a + b = 1. Prove that
\begin{align*}
a^a b^b + a^b b^a \leq 1.
\end{align*}

\item Let a and b be real numbers such that a $\neq$ 0. Prove that not all that roots of 
\begin{align}
ax^4 + bx^3 + x^2 + x + 1 = 0
\end{align} 
can be real.

\item Let ABC be an acute-angle triangle. The circle $\Gamma$ with BC as diameter intersects AB and AC again at P and Q,respectively. Determine $\angle$BAC given that the orthocentre of triangle APQ lies on $\Gamma$.

\item Let ABC be a triangle with $\angle$A = $90^{o}$ and AB = AC. Let D and E be points on the segement BC such that BD:DE:EC=3:5:4 Prove that $\angle$DAE = $45^{o}$.

\item Let
\begin{align} 
P_1(x) = x^2 + a_1 + b_1  
\end{align}
and
\begin{align} 
P_2(x) = x^2 + a_2x + b_2
\end{align} 
be two quadratic polynomials with integer coefficients. Suppose $a_1 \neq a_2$ and there exist an integer 
m $\neq$ n such that $P_1$(m) = $P_2$(n), $P_2$(m) = $P_1$(n). Prove that $a_1-a_2$ is even.

\item Find real numbers such that 3 $<$ a $<$ 4 and $a(a-3\{a\})$ is an integer. 
\{Here \{a\} denotes the fractional part of a. For example \{1, 5\} = 0.5; \{-3, 4\} = 0.6.\}

\item Let ABC be a right-angled triangle with $\angle$b = $90^{o}$. Let I be the incentre of ABC. Draw a line  perpendicular to AI at I. Let it intersect the line CB at D. Prove that CI is perpendicular to AD prove that 
ID = $\sqrt{b(b - a)}$ where BC=a and CA=b.

\item Let ABC be a triangle with centroid G. Let the circumcircle of triangle AGB intersect the line BC in X different from B;and the ciecumcircle of triangle AGC intersect the line BC in Y different from C. Prove that G is the centroid of triangle AXY.


\item Let a, b, c be positive real numbers such that
\begin{align*}
\frac{a}{1+a} + \frac{b}{1+b} + \frac{c}{1+c} = 1
\end{align*}
Prove that abc $\leq$ $\frac{1}{8}$.

\item Let a, b, c, d, e, f be positive integers such that 
\begin{align*}
\frac{a}{b} < \frac{c}{d} < \frac{e}{f}.
\end{align*}
Suppose af - be = -1. Show that d $\geq$ b + f.

\item There are 100 countries participating in an olympiad. Suppose n is a positive integer such that each of the countries is willing to communicate in exactly n languages. If each set of 20 countries can communicate in at least one common language,and no language is common to all 100 countries, what is the minimum possible value of n?

\item 
\begin{enumerate}
\item Given any natural number N, prove that there exists a strictly increasing sequence of N positive integers in harmonic progression.
\item Prove that there cannot exist a strictly increasing infinite sequence of positive integers which is in harmonic progression.
\end{enumerate}

\item Let a, b, c be positive real number such that
\begin{align*}
\frac{ac}{1+bc} + \frac{bc}{1+ca} + \frac{ca}{1+ab} = 1.
\end{align*}
Prove that 
\begin{align*}
\frac{1}{a^3} + \frac{1}{b^3} + \frac{1}{c^3} \geq 6\sqrt{2}
\end{align*}

\item Consider $n^{2}$ units squares in the xy-plane centred at point (i, j) with integer co-ordinates, 
1 $\leq$ i $\leq$ n, 1 $\leq$ j $\leq$ n. It is required to colour each unit square in such a way that when ever 
1 $\leq$ i $<$ j $\leq$ n and 1 $\leq$ k $<$ l $\leq$ n,the three squares with centres at (i, k),(j, k),(j, l)have distinct colours. What is the least possible number of colours needed?

\item Let x, y, z be real numbers each greater than 1. Prove that 
\begin{align*}
\frac{x+1}{y+1} + \frac{y+1}{z+1} + \frac{z+1}{x+1} \leq \frac{y-1}{z-1} + \frac{z-1}{x-1}.
\end{align*}

\end{enumerate}

\end{document}


